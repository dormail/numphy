\documentclass[11pt a4paper]{article}
\usepackage[margin=2cm]{geometry}
\usepackage{amsmath, amssymb}
\usepackage{graphicx}
\usepackage{float}
\usepackage{aligned-overset}
\usepackage{subcaption}

% partielle ableitungen
\newcommand{\delr}{\partial_r}
\newcommand{\deltheta}{\partial_\theta}
\newcommand{\delphi}{\partial_\varphi}

% elektrische feldkonstante
\newcommand{\epsz}{\epsilon_0}
% 1 / 4pi eps
\newcommand{\kco}{\frac{1}{4\pi\epsilon_0}}

% fancy header
\usepackage{fancyhdr}
\fancyhf{}
% vspaces in den headern fuer Distanzen notwendig
% linke Seite: Namen der Abgabegruppe
\lhead{\textbf{Tahir Kamcili \\ Alfredo Manente \\ Luca Dietrich \\ Matthias Maile}\vspace{1.5cm}}
% rechte Seite: Modul, Gruppe, Semester
\rhead{\textbf{Numerische Mathematik für \\ Physiker und Ingenieure\\Sommersemester 2020}\vspace{1.5cm}}
% Center: nr. des blattes
\chead{\vspace{2.5cm}\huge{\textbf{4. Übungszettel}}}
% benoetigt damit der eigentliche Text nicht in der Überschrift steckt

% zum zeichnen tikz
\usepackage{tikz}

% fuer fabigen text
\usepackage{xcolor}

\begin{document}
\section*{Aufgabe 4.2}
a)
\begin{align*}
	P_4 \ni p(x) &= ax^4 + bx^3 + cx^2 + dx + e \\
	\Rightarrow 
	\int_{-1}^1 p(x) dx 
	&= \int_{-1}^1 \left(ax^4 + bx^3 + cx^2 + dx + e\right) dx \\
	% stammfunktion bilden
	&= \left[\frac a5 x^5 + \frac b4 x^4 + \frac c3 x^3 + 
	\frac d2 x + ex + f \right]_{-1}^1 \\
	% einsetzen
	&= \frac a5 1^5 + \frac b4 1^4 + \frac c3 1^3 + 
	\frac d2 1^2 + e \cdot 1 + f - \left( 
	\frac a5 (-1)^5 + \frac b4 (-1)^4 + \frac c3 (-1)^3 + 
	\frac d2 (-1)i^2 + e \cdot (-1) + f \right) \\
	% zur kuerzung umformen
	&= \underbrace{\frac a5 + \frac a5}_{\frac{2a}{5}}
	+ \underbrace{\frac b4 - \frac b4 }_{=0}
	+ \underbrace{\frac c3  + \frac c3}_{\frac{2c}{3}}
	+ \underbrace{\frac d2 - \frac  d2}_{=0}
	+ \underbrace{e + e }_{2e}
	+ \underbrace{f - f}_{=0} \\
	% Kuerzen
	&= \frac{2a}5 + \frac{2c}3 + 2e \\
	% interpolations formel
	\Rightarrow I_h(p_4) 
	&= \frac19 \left( p_4(-1) + 8 p_4(-0.5) + 8 p_4(0.5) + p(1) \right)
	\\
	% einsetzen in formel fuer p
	&= \frac19 \left( a - b + c - d + e \right)
	+ \frac19 \left(\frac a{2} - b + 2c 
	- 4d + 8e\right) 
	+ \frac19 \left(\frac a{2} + b + 2c + 4d + 8e\right) \\
	&+ \frac19 \left(a + b + c + d + e \right) \\
	% kuerzen 
	&= \frac29 a + \frac29 c + \frac29 e+ \frac a9 + \frac{16c}{9}
	+ \frac{8e}{9} \\
	% weiter kuerzen
	&= \frac a3  + \frac {2c}{3} + 2e \\
	% vergleiche
	\Rightarrow I_h(p_4) &= \int_{-1}^1 p_4(x) \ dx 
	- \frac{2a}{5} + \frac a3
\end{align*}
Aus dem Ergebnis wird deutlich, dass der Interpolationsfehler nur vom
Leitkoeffizienten $a$ abhängt. Wenn $a=0$ wäre, würde der Fehler ebenfalls
0 sein. Dann wäre allerdings $p_4 \notin P_4$.\\
Somit wird mit der Formel kein $p_4 \in P_4$ präzise integriert.\\
(Falls nur ein Polynom gesucht war: $p_4 = x^4$ wird nicht genau 
integriert)
\newline
\vspace{0.5cm}
\newline
b) Wenn wir das Ergebnis aus a) angucken, lässt 
sich erkennen, dass der Fehler 
nur von $a$ abhängt:
\[
	I_h(p) - \int_{-1}^1 p(x) dx= \frac a3 - \frac{2a}{5}
\]
Da dieser Koeffizient bei einem $p \in P_3$ verschwinden muss, wird 
jedes Polynom 3. Grades (oder niedriger) genau Integriert.
\newline
\vspace{0.5cm}
\newline
c) als Funktion kann man die Normalverteilung nehmen, wobei $\mu$ dem 
Stützpunkt $x_j$ entspricht zu dem das Gewicht $a_j < 0$ gehört.
\newline
Wenn man dann die Standartabweichung $\sigma$ gegen 0 laufen lässt, geht
gehen die Werte $f(x_k)_{k\neq j} \rightarrow 0$. Die Quadraturformel 
verinfacht sich dann:
\[
	I_A^{(n)} (f) = \sum_{k=0}^n a_k f(x_k) \approx
	a_0 \cdot 0 + a_1 \cdot 0 + \hdots + a_j \cdot f(x_j) + \hdots =
	a_j \cdot f(x_j) < 0
\]
Das Integral über die Normalverteilung ist natürlich immer positiv.


\end{document}
